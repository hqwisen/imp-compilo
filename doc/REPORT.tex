\documentclass[letterpaper]{article}
\usepackage{natbib}
\usepackage[utf8]{inputenc}
\usepackage[margin=1.5in]{geometry}



\title{Introduction to language theory and compiling
\\ Project - Part 1}
\author{Université Libre de Bruxelles \\
\\ Hakim Boulahya}


\begin{document}
\maketitle

\section{Definition}
\begin{itemize}
    \item MUST
    \item OR
    \item CAN
\end{itemize}

\begin{itemize}
    \item case-senstive characters
    \item alpha numeric
\end{itemize}

\section{Extended regular expressions}
This section describe extended regular expressions use by the lexer.
The shortcuts section describes simple ERE that are used to write
the advanced ERE.
\subsection{Shortcuts}

\begin{enumerate}
    \item \texttt{AlphaUpperCase = [A-Z]} : Match all uppercase letters
    \item \texttt{AlphaLowerCase = [a-z]} : Match all lowercase letters
    \item \texttt{Alpha = \{AlphaUpperCase\}|\{AlphaLowerCase\}} :
    Match alphabet case-sensitive characters
    \item \texttt{Numeric = [0-9]} : Match digit characters
    \item \texttt{AlphaNumeric = \{Alpha\}|\{Numeric\}} :
    Match alphabetic \textbf{OR} digit characters
\end{enumerate}

\subsection{Advanced}

\begin{enumerate}
    \item \texttt{Number = ([1-9]{Numeric}*)|0} :
    Match all numbers that \textbf{MUST} start with a non-zero digit and follow
    by 0 or more digits (first RE),
    \textbf{OR} match only zero digit (second RE). This RE correspond to
    the [Number] lexical unit.
    \item \texttt{VarName = {Alpha}{AlphaNumeric}*}
    Match identifiers. Identifiers \textbf{MUST} starts with a alphabetic
    character and \textbf{CAN} be followed by 0 or more digits \textbf{OR}
    alphabetic characters.
    Correspond to the [VarName] lexical unit.
    \item \texttt{Spaces = \\s+}
\end{enumerate}


\section{All in yyinitial vs states for different grammars}

\end{document}
