\documentclass[letterpaper]{article}
\usepackage{natbib}
\usepackage[utf8]{inputenc}
\usepackage[margin=1.5in]{geometry}
\usepackage{listings}

\title{Introduction to language theory and compiling
\\ Project - Part 1}
\author{Université Libre de Bruxelles \\
\\ Hakim Boulahya}


\begin{document}
\maketitle
\tableofcontents
\section{Definition}

\subsection{Emphasizing}

\begin{itemize}
    \item MUST
    \item OR
    \item CAN
\end{itemize}

\subsection{Notation and defintions}

\begin{itemize}
    \item case-senstive characters
    \item alpha numeric
    \item ERE: Extended Regular Expression
\end{itemize}

\section{Extended Regular Expressions}
This section describe extended regular expressions use by the lexer.
The shortcuts section describes simple ERE that are used to write
the advanced ERE.
\subsection{Shortcuts}

\begin{enumerate}
    \item \texttt{AlphaUpperCase = [A-Z]} : Match all uppercase letters
    \item \texttt{AlphaLowerCase = [a-z]} : Match all lowercase letters
    \item \texttt{Alpha = \{AlphaUpperCase\}|\{AlphaLowerCase\}} :
    Match alphabet case-sensitive characters
    \item \texttt{Numeric = [0-9]} : Match digit characters
    \item \texttt{AlphaNumeric = \{Alpha\}|\{Numeric\}} :
    Match alphabetic \textbf{OR} digit characters
\end{enumerate}

\subsection{Advanced}
\label{advanced-ere}
\begin{enumerate}
    \item \texttt{Number = ([1-9]{Numeric}*)|0} :
    Match all numbers that \textbf{MUST} start with a non-zero digit and follow
    by 0 or more digits (first RE),
    \textbf{OR} match only zero digit (second RE). This RE correspond to
    the [Number] lexical unit.
    \item \texttt{VarName = {Alpha}{AlphaNumeric}*}
    Match identifiers. Identifiers \textbf{MUST} starts with a alphabetic
    character and \textbf{CAN} be followed by 0 or more digits \textbf{OR}
    alphabetic characters.
    Correspond to the [VarName] lexical unit.
    \item \texttt{Blank = \\s+}
\end{enumerate}

\subsection{Keyword}

\label{ere-keywords}

All the following keyword are matching \textit{themself} directly: \\

\begin{tabular}{|c|c|c|c|c|}
    \hline
    ; & := & ( & )  & + \\
    \hline
    - & *  & / & if & then \\
    \hline
    endif & else & while & for & from \\
    \hline
    by & to & do & done & note \\
    \hline
    and & or & = & $>$= & $>$ \\
    \hline
    $<$= & $<$ & $<>$ & print & read \\
    \hline
    begin & end & & &  \\
    \hline
\end{tabular}



\section{Implementation}

\subsection{Code structure}

\paragraph{Files}

Our implementation is contained in two files: \texttt{LexicalAnalyzer.flex},
contains the ERE that analyse the input sources and \texttt{ImpCompilo.java}
that contains the Java source code, used in the regular Expressions
matching actions.


\paragraph{Code flexibility}

To allow a better code flexibility, the JFlex generated class will
\texttt{extends} \texttt{ImpCompilo} class. This allow to develop the
compilator directly from a Java source class, instead of putting the
ERE and the logical code inside the .flex file. The generated class is named
\texttt{Main.java}, which should be used to run the scanner.

\subsection{Keywords}

\paragraph{}

This section describe the states\cite{jflexdocstates} implemented in our
lexical analyzer and the corresponding ERE used.

\paragraph{}

One state is enough (necessary) to return the IMP language tokens.
The state \textbf{YYINITIAL} (the default inclusive state) is used to match
all the keywords listed in section \ref{ere-keywords}. For each keyword
the action is running the method \texttt{ImpCompilo.symbol(LexicalUnit)}.
This method runs the following instruction:
\begin{enumerate}
    \item if LexicalUnit is VarName: add it to \texttt{identifiers} dictionnary
    \item Add the symbol to \texttt{symbols}
    \item Output the symbol using the provided \texttt{toString} method.
\end{enumerate}
\paragraph{Example}

For the keyword \texttt{if}, the corresping ERE line in YYINITIAL is:

\begin{lstlisting}[frame=single]
"if"             {return symbol(LexicalUnit.IF);}
\end{lstlisting}

\paragraph{}
[VarName] and [Number] lexical units are also handled in \textbf{YYINITIAL}.
The only difference is that the right-side of the line code is not a keyword
but the correspding ERE defined in section \ref{advanced-ere}.

\subsection{Blank characters}

Blank characters, space, new line and tabs characters,
are ignored by the scanner. It is
done by matching space \textit{i.e.} the ERE \texttt{\s+}, which match
1 or more blank character, to an empty action. We have in YYINITIAL:
\begin{lstlisting}[frame=single]
{Blank}        {}
\end{lstlisting}

\subsection{Unknown token}

If some characters don't match any of the ERE in YYINITIAL, the scanner
will output:

\begin{lstlisting}[frame=single]
Unknown token: '<text>'
\end{lstlisting}

Note that this does not append in the COMMENT state (section \ref{comment}),
since everything is ignored.

\subsection{Comments}
\label{comment}

\paragraph{}

Comments can only be represented by an enclosed text:
 \texttt{(* \textit{Comments content} *)}.

Therefore the solution provide to ignore the comments is the following:
\begin{itemize}

\item In YYINITIAL and comment is opened, i.e. matches \texttt{(*},
change state to COMMENT

\item In COMMENT, Ignore all characters
(including blank characters) except the ending comment
characters \texttt{*)}.
If these characters are matched change state to YYINITIAL.

\end{itemize}

This process will ignore all characters in enclosed in comments.

\subsubsection{Forbidden behaviours}
\paragraph{}

There are two forbidden behaviours of comment detection that can be highlighted:

\begin{enumerate}
    \item A comment is opened but not closed.
    \item A closed comment is detected, but no opening precede it.
\end{enumerate}

The behaviour number 1 will ignore every characters until the EOF, since
the state will change to COMMENT.
\paragraph{}

The behaviour number 2 is in state YYINITIAL, which means that the \textit{closed}
comment characters \texttt{*)} will match
the TIMES and LPAREN \textt{LexicalUnit},
because the closed comment characters are only defined in COMMENT state.
\paragraph{}

With our implementation the nested comments, that are forbidden, will
raise a syntax error, when detected in the parser. Because with
\texttt{(* (* *) *)}, the second comment opening will be ignored in the
COMMENT state, the first closed comment characters will close the first comment
opening. The last closed comment characters is left as if we were in
behaviour number 2.

\subsubsection{Nested comments (Bonus)}

Use counter ?

\section{All in yyinitial vs states for different grammars}

Changes between states are diffuclt to detect Unknown tokens
it increase diffulcty of impl
better to use one state
always makes no sense to use multi state since the grammar is checked
during the scanner, therefore the grammar should be impl in part 2.

\end{document}
