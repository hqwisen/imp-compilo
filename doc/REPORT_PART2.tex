\documentclass[letterpaper]{article}
\usepackage{natbib}
\usepackage[utf8]{inputenc}
\usepackage[margin=3.5cm]{geometry}
\usepackage{listings}
\usepackage{adjustbox}
\usepackage{xcolor}
\usepackage{verbatim}
\usepackage{graphicx}%images
\usepackage{fancyhdr}%for headers and footers
\usepackage{adjustbox}
\usepackage{verbatim}
\usepackage{listings}
\usepackage{fancyhdr}
\usepackage{multirow}
\usepackage{amsmath}
\usepackage{array}
\usepackage{mathpazo}
\usepackage{subcaption}
\usepackage{float}
\usepackage{csvsimple}
\usepackage{filecontents}
\usepackage{lscape}
\usepackage{afterpage}
\usepackage{hyperref}
\usepackage{inconsolata}
\usepackage{color}

\definecolor{pblue}{rgb}{0.13,0.13,1}
\definecolor{pgreen}{rgb}{0,0.5,0}
\definecolor{pred}{rgb}{0.9,0,0}
\definecolor{pgrey}{rgb}{0.46,0.45,0.48}

\lstset{language=Java,
  showspaces=false,
  showtabs=false,
  breaklines=true,
  showstringspaces=false,
  breakatwhitespace=true,
  commentstyle=\color{pgreen},
  keywordstyle=\color{pblue},
  stringstyle=\color{pred},
  basicstyle=\ttfamily,
  moredelim=[il][\textcolor{pgrey}]{$ $},
  moredelim=[is][\textcolor{pgrey}]{\%\%}{\%\%}
}



\hypersetup{
    colorlinks,
    citecolor=black,
    filecolor=black,
    linkcolor=black,
    urlcolor=black
}


% ------ HEADERS AND FOOTERS -----------
% \lhead{INFO-F403}
% \rhead{Project Report - Part 1}
%\pagestyle{fancy}
% \rfoot{\thepage}
%\cfoot{}
%\lfoot{Academic Year 2017-18}

\newcommand{\HRule}{\rule{\linewidth}{0.5mm}} %newcommand for cover page

\begin{document}
\begin{titlepage}
\begin{center}


\textsc{\LARGE universit\'e libre de bruxelles}\\[1.0cm]
\textsc{\Large D\'epartment d'Informatique}\\[1.5cm]

% Upper part of the page. The '~' is needed because \\
% only works if a paragraph has started.
\includegraphics[width=0.3\textwidth]{image/ulblogo.jpg}~\\[1cm]

\textsc{
\large INFO-F403 \\
\Large  Introduction to language theory and compiling
 \\[1cm]}
% Title
\HRule \\[0.7cm]

{ \huge \bfseries Project Report – Part 2  \\[0.7cm] }

\HRule \\[2cm]

% Author and supervisor
\noindent
\begin{center} \large

\emph{Author:}\\
\Large Hakim \textsc{Boulahya}\\
\end{center}
\begin{center} \large

% \emph{Professor:} \\
% Gilles \textsc{Geeraerts}\\

\end{center}

\vfill

% Bottom of the page
{\large \today}

\end{center}
\end{titlepage}

\tableofcontents
\newpage

\section{Introduction}

\section{Grammar}

\paragraph{Unproductive/unreachable symbols}

In the given grammar, there is no unproductive and/or unreachable symbols.

\subsection{Priority and associativity of the operators}

\paragraph{Notation} In this section, P\&A refers to priority and associativity
of the operators, AE to arithmetic expression and BE to boolean expression.

\paragraph{}

An AE must always be \textit{processed}
before being compared
to another AE in a BE. Logically, we will consider first to unambiguous
the arithmetic expressions and then the boolean expressions.

\subsubsection{Arithmetic expressions}

\paragraph{}

Let's consider the P\&A of arithmetic expressions in Figure \ref{fig:AE-pa}
and the initial grammar in Figure \ref{fig:AE-initial}.

\begin{figure}[H]
    \centering
    \begin{tabular}{|c|c|}
        \hline
        Operators & Associativity \\
        \hline
        \hline
        - & right \\
        \hline
        *, / & left \\
        \hline
        +, - & left \\
        \hline
    \end{tabular}
    \caption{Priority and associativity of AE}
    \label{fig:AE-pa}
\end{figure}

\begin{figure}[H]
        \centering
        \begin{tabular}{l l}
            $<$ExprArith$>$ &$\rightarrow$ [VarName] \\
             &$\rightarrow$ [Number] \\
             &$\rightarrow$ ($<$ExprArith$>$) \\
             &$\rightarrow$ - $<$ExprArith$>$ \\
             &$\rightarrow$
            $<$ExprArith$>$ $<$Op$>$ $<$ExprArith$>$ \\
            $<$Op$>$ &$\rightarrow$ $+$ \\
             &$\rightarrow$ $-$ \\
             &$\rightarrow$ $*$ \\
             &$\rightarrow$ $/$ \\
        \end{tabular}
        \caption{Initial grammar of AE}
        \label{fig:AE-initial}
\end{figure}

As mention in the course page 111, an AE must be a \textit{sum of products},
more specifically in our case a \textit{\{sum, substraction\}
of \{produts, division\}}. We will use the same atom definition
 as in the course,
with \texttt{[Number]} as the constant rule and \texttt{[VarName]}
as the id rule.
The minus operator as a right associativity means that it is always linked
to the atom next to it, so we will set this operator directly as
an atom rule.
Same thing goes for the parenthesis, it should be handled without considering
the operators outside the parenthesis, so as an atom.
Figure \ref{fig:AE-unambi} show the unambiguous grammar of AE.

\begin{figure}[H]
    \centering
    \begin{tabular}{l l}
        $<$ExprArith$>$ &$\rightarrow$
        $<$ExprArith$>$ $<$SumSubOp$>$ $<$ExprProd$>$ \\
         &$\rightarrow$ $<$ExprProd$>$ \\
        $<$ExprProd$>$ &$\rightarrow$
        $<$ExprProd$>$ $<$ProdOp$>$ $<$Atom$>$ \\
         &$\rightarrow$ $<$Atom$>$ \\

        $<$SumSubOp$>$ &$\rightarrow$ + \\
         &$\rightarrow$ - \\

        $<$ProdOp$>$ &$\rightarrow$ * \\
         &$\rightarrow$ / \\

        $<$Atom$>$ &$\rightarrow$ [VarName] \\
         &$\rightarrow$ [Number] \\
         &$\rightarrow$ - $<$Atom$>$ \\
         &$\rightarrow$ ($<$ExprArith$>$) \\
    \end{tabular}
    \caption{Unambiguous grammar of AE}
    \label{fig:AE-unambi}
\end{figure}


\subsubsection{Boolean expressions}

Let's consider the P\&A of boolean expressions in Figure \ref{fig:BE-pa}
and the initial grammar in Figure \ref{fig:BE-initial}.

\begin{figure}[H]
    \centering
    \begin{tabular}{|c|c|}
        \hline
        Operators & Associativity \\
        \hline
        \hline
        not & right \\
        \hline
        $>$, $<$, $>=$, $<=$, $=$, $<>$ / & left \\
        \hline
        and & left \\
        \hline
        or & left \\
        \hline
    \end{tabular}
    \caption{Priority and associativity of BE}
    \label{fig:BE-pa}
\end{figure}

\begin{figure}[H]
    \centering
    \begin{tabular}{l l}
        $<$Cond$>$ & $\rightarrow$ $<$Cond$>$ $<$BinOp$>$ $<$Cond$>$ \\
         & $\rightarrow$ \texttt{not} $<$SimpleCond$>$\\
         & $\rightarrow$ $<$SimpleCond$>$\\
        $<$SimpleCond$>$ & $\rightarrow$
        $<$ExprArith$>$ $<$Comp$>$ $<$ExprArith$>$ \\
        $<$BinOp$>$ & $\rightarrow$ \texttt{and}\\
         & $\rightarrow$ \texttt{or}\\
        $<$Comp$>$ & $\rightarrow$ $=$\\
         & $\rightarrow$ $>=$\\
         & $\rightarrow$ $>$\\
         & $\rightarrow$ $<=$\\
         & $\rightarrow$ $<$\\
         & $\rightarrow$ $<>$\\
    \end{tabular}
    \caption{Initial grammar of BE}
    \label{fig:BE-initial}

\end{figure}

Following the same principle as for AE, we have here
\textit{disjonction of conjonctions}. Figure \ref{fig:BE-unambi}
shows the unambiguous grammar of BE.


\begin{figure}[H]
    \centering
    \begin{tabular}{l l}
        $<$Cond$>$ & $\rightarrow$ $<$Cond$>$ \texttt{or} $<$ConjCond$>$ \\
         & $\rightarrow$ $<$ConjCond$>$\\

        $<$ConjCond$>$ & $\rightarrow$
        $<$ConjCond$>$ \texttt{and} $<$AtomCond$>$\\
         & $\rightarrow$ $<$AtomCond$>$\\

        $<$AtomCond$>$ & $\rightarrow$ $<$SimpleCond$>$\\
         & $\rightarrow$ \texttt{not} $<$SimpleCond$>$\\

        $<$SimpleCond$>$ & $\rightarrow$
        $<$ExprArith$>$ $<$Comp$>$ $<$ExprArith$>$\\

        $<$Comp$>$ & $\rightarrow$ $=$\\
         & $\rightarrow$ $>=$\\
         & $\rightarrow$ $>$\\
         & $\rightarrow$ $<=$\\
         & $\rightarrow$ $<$\\
         & $\rightarrow$ $<>$\\
    \end{tabular}
    \caption{Unambiguous grammar of BE}
    \label{fig:BE-unambi}

\end{figure}


\subsection{Removing left recusion}
\paragraph{}

The only rules where left-recursion appears are \texttt{$<$ExprArith$>$},
\texttt{$<$ExprProd$>$}, \texttt{$<$Cond$>$} and \texttt{$<$ConjCond$>$}
variables
that are in the unambiguous grammar.
We have to remove the left-recursion for the unambiguous grammar
in Figure \ref{fig:AE-unambi} and \ref{fig:BE-unambi}. The transformed
grammars are shown in Figure \ref{fig:AE-noleft} and \ref{fig:BE-noleft}.
To enhance readability of new introduced rules,
we used the word \textit{Prime} instead of the punctuation mark.

Those are the only rules where left-recusion appears (including indirect
left-recursion), all the other rules
are right-recusion or no recursion at all.
\begin{figure}[H]
    \centering
    \begin{tabular}{l l}
        $<$ExprArith$>$ & $\rightarrow$
        $<$ExprProd$>$ $<$ExprArithPrime$>$ \\

        $<$ExprArithPrime$>$ & $\rightarrow$
        $<$SumSubOp$>$ $<$ExprProd$>$ $<$ExprArithPrime$>$ \\
         & $\rightarrow$ $\epsilon$\\


        $<$ExprProd$>$ & $\rightarrow$ $<$Atom$>$ $<$ExprProdPrime$>$\\

        $<$ExprProdPrime$>$ & $\rightarrow$
        $<$ProdOp$>$ $<$Atom$>$ $<$ExprProdPrime$>$\\
         & $\rightarrow$ $\epsilon$\\


        $<$SumSubOp$>$ & $\rightarrow$ $+$\\
         & $\rightarrow$ $-$\\

        $<$ProdOp$>$ & $\rightarrow$ $*$\\
         & $\rightarrow$ $/$\\

        $<$Atom$>$ & $\rightarrow$ [VarName]\\
         & $\rightarrow$ [Number]\\
         & $\rightarrow$ - $<$Atom$>$\\
         & $\rightarrow$ ($<$ExprArith$>$)\\

    \end{tabular}
    \caption{Left-recursion removed for AE grammar}
    \label{fig:AE-noleft}
\end{figure}

\begin{figure}[H]
    \centering
    \begin{tabular}{l l}
        $<$Cond$>$ & $\rightarrow$ $<$ConjCond$>$ $<$CondPrime$>$ \\


        $<$CondPrime$>$ & $\rightarrow$
        \texttt{or} $<$ConjCond$>$ $<$CondPrime$>$ \\
         & $\rightarrow$ $\epsilon$ \\

        $<$ConjCond$>$ & $\rightarrow$ $<$AtomCond$>$ $<$ConjCondPrime$>$ \\


        $<$ConjCondPrime$>$ & $\rightarrow$
        and $<$AtomCond$>$ $<$ConjCondPrime$>$ \\
         & $\rightarrow$ $\epsilon$\\

        $<$AtomCond$>$ & $\rightarrow$ $<$SimpleCond$>$\\
         & $\rightarrow$ \texttt{not} $<$SimpleCond$>$\\

        $<$SimpleCond$>$ & $\rightarrow$
        $<$ExprArith$>$ $<$Comp$>$ $<$ExprArith$>$\\

        $<$Comp$>$ & $\rightarrow$ =\\
         & $\rightarrow$ $>=$\\
         & $\rightarrow$ $>$\\
         & $\rightarrow$ $<=$\\
         & $\rightarrow$ $<$\\
         & $\rightarrow$ $<>$\\

    \end{tabular}
    \caption{Left-recursion removed for BE grammar}
    \label{fig:BE-noleft}
\end{figure}

\subsection{Factorisation}

\paragraph{}

We can only factorize the following variables:
\texttt{$<$InstList$>$}, \texttt{$<$If$>$} and \texttt{$<$For$>$}.
Figure \ref{fig:facto} shows the factorized rules.

\begin{figure}[H]
    \centering
    \begin{tabular}{l l}
        $<$InstList$>$ & $\rightarrow$ $<$Instruction$>$ $<$InstListSeq$>$\\
        $<$InstListSeq$>$ & $\rightarrow$ ; $<$InstList$>$\\
         & $\rightarrow$ $\epsilon$\\

        $<$If$>$ & $\rightarrow$
        \texttt{if} $<$Cond$>$ \texttt{then} $<$Code$>$ $<$IfSeq$>$\\
        $<$IfSeq$>$ & $\rightarrow$ \texttt{endif}\\
         & $\rightarrow$
         \texttt{else} $<$Code$>$ \texttt{endif}\\

        $<$For$>$ & $\rightarrow$
        \texttt{for} [VarName] \texttt{from}
        $<$ExprArith$>$ $<$ForOp$>$ \\
        &$\quad$\texttt{to}
        $<$ExprArith$>$ \texttt{do} $<$Code$>$ \texttt{done}\\

        $<$ForOp$>$ & $\rightarrow$ \texttt{by} $<$ExprArith$>$\\
         & $\rightarrow$ $\epsilon$\\

    \end{tabular}
    \caption{Factorized rules}
    \label{fig:facto}
\end{figure}

\subsection{Transformed Grammar}

\paragraph{}

The complete transformed grammar is shown in Figure \ref{fig:fullcfg}.

\afterpage{
\newgeometry{margin=2.5cm}
\begin{figure}[H]
    \centering
    \begin{tabular}{r l l}
        $[1]$ & $<$Program$>$ & $\rightarrow$ \texttt{begin} $<$Code$>$
        \texttt{end} \\

        $[2]$ & $<$Code$>$ & $\rightarrow$ $<$InstList$>$ \\
        $[3]$ & & $\rightarrow$ $\epsilon$ \\

        $[4]$ & $<$InstList$>$ & $\rightarrow$ $<$Instruction$>$
        $<$InstListSeq$>$ \\

        $[5]$ & $<$InstListSeq$>$ & $\rightarrow$ ; $<$InstList$>$ \\
        $[6]$ & & $\rightarrow$ $\epsilon$ \\

        $[7]$ & $<$Instruction$>$ & $\rightarrow$ $<$Assign$>$ \\
        $[8]$ &                   & $\rightarrow$ $<$If$>$ \\
        $[9]$ &                   & $\rightarrow$ $<$While$>$ \\
        $[10]$ &                   & $\rightarrow$ $<$For$>$ \\
        $[11]$ &                   & $\rightarrow$ $<$Print$>$ \\
        $[12]$ &                   & $\rightarrow$ $<$Read$>$ \\

        $[13]$ & $<$Assign$>$ & $\rightarrow$
         [VarName] := $<$ExprArith$>$  \\

        $[14]$ & $<$ExprArith$>$ & $\rightarrow$
        $<$ExprProd$>$ $<$ExprArithPrime$>$ \\

        $[15]$ & $<$ExprArithPrime$>$ & $\rightarrow$
        $<$SumSubOp$>$ $<$ExprProd$>$ $<$ExprArithPrime$>$ \\
        $[16]$ & & $\rightarrow$ $\epsilon$ \\

        $[17]$ & $<$ExprProd$>$ & $\rightarrow$
        $<$Atom$>$ $<$ExprProdPrime$>$ \\

        $[18]$ & $<$ExprProdPrime$>$ & $\rightarrow$
        $<$ProdOp$>$ $<$Atom$>$ $<$ExprProdPrime$>$ \\
        $[19]$ & & $\rightarrow$ $\epsilon$ \\

        $[20]$ & $<$SumSubOp$>$ & $\rightarrow$ $+$ \\
        $[21]$ & & $\rightarrow$ $-$ \\
        $[22]$ & $<$ProdOp$>$ & $\rightarrow$ $*$ \\
        $[23]$ & & $\rightarrow$ $/$ \\
        $[24]$ & $<$Atom$>$ & $\rightarrow$ [VarName] \\
        $[25]$ & & $\rightarrow$ [Number] \\
        $[26]$ & & $\rightarrow$ - $<$Atom$>$ \\
        $[27]$ & & $\rightarrow$ ($<$ExprArith$>$) \\

        $[28]$ & $<$If$>$ & $\rightarrow$
        \texttt{if} $<$Cond$>$ \texttt{then} $<$Code$>$ $<$IfSeq$>$ \\
        $[29]$ & $<$IfSeq$>$ & $\rightarrow$
        \texttt{endif} \\
        $[30]$ & & $\rightarrow$
        \texttt{else} $<$Code$>$ \texttt{endif} \\

        $[31]$ & $<$Cond$>$ & $\rightarrow$
        $<$ConjCond$>$ $<$CondPrime$>$ \\
        $[32]$ & $<$CondPrime$>$ & $\rightarrow$
        \texttt{or} $<$ConjCond$>$ $<$CondPrime$>$ \\
        $[33]$ & & $\rightarrow$ $\epsilon$ \\
        $[34]$ & $<$ConjCond$>$ & $\rightarrow$
        $<$AtomCond$>$ $<$ConjCondPrime$>$ \\
        $[35]$ & $<$ConjCondPrime$>$ & $\rightarrow$
        \texttt{and} $<$AtomCond$>$ $<$ConjCondPrime$>$ \\
        $[36]$ & & $\rightarrow$ $\epsilon$ \\
        $[37]$ & $<$AtomCond$>$ & $\rightarrow$ $<$SimpleCond$>$ \\
        $[38]$ & & $\rightarrow$ \texttt{not} $<$SimpleCond$>$ \\

        $[39]$ & $<$SimpleCond$>$ & $\rightarrow$
        $<$ExprArith$>$ $<$Comp$>$ $<$ExprArith$>$ \\
        $[40]$ & $<$Comp$>$ & $\rightarrow$ $=$ \\
        $[41]$ & & $\rightarrow$ $>=$ \\
        $[42]$ & & $\rightarrow$ $>$ \\
        $[43]$ & & $\rightarrow$ $<=$ \\
        $[44]$ & & $\rightarrow$ $<$ \\
        $[45]$ & & $\rightarrow$ $<>$ \\

        $[46]$ & $<$While$>$ & $\rightarrow$
        \texttt{while} $<$Cond$>$ \texttt{do}
        $<$Code$>$ \texttt{done} \\
        $[47]$ & $<$For$>$ & $\rightarrow$
        \texttt{for} [VarName] \texttt{from}
        $<$ExprArith$>$ $<$ForOp$>$ \\
        & &
        $\quad$ \texttt{to} $<$ExprArith$>$
        \texttt{do} $<$Code$>$ \texttt{done} \\
        $[48]$ & $<$ForOp$>$ & $\rightarrow$
        \texttt{by} $<$ExprArith$>$ \\
        $[49]$ & & $\rightarrow$ $\epsilon$ \\
        $[50]$ & $<$Print$>$ & $\rightarrow$
        \texttt{print}([VarName]) \\
        $[51]$ & $<$Read$>$ & $\rightarrow$ \texttt{read}([VarName]) \\
    \end{tabular}
    \caption{Transformed Grammar}
    \label{fig:fullcfg}
\end{figure}
\clearpage
\restoregeometry
}

\section{LL(1) parser}

\subsection{First$^1(\cdot)$ and Follow$^1(\cdot)$}
\afterpage{
\newgeometry{margin=3cm}
\begin{figure}[H]
    \centering
    \renewcommand{\arraystretch}{1.5}
    \begin{tabular}{| r || c | c |}
        \hline
        \textbf{Variables} & \textbf{First$^1(\cdot)$}
        & \textbf{Follow$^1(\cdot)$} \\
        \hline
        \hline
        Program & begin & \\
        \hline
        Code & [VarName] if while for print read $\epsilon$ &
        end else endif done \\
        \hline
        InstList & [VarName] if while for print read &
        end else endif done \\
        \hline
        InstListSeq & ; $\epsilon$ &
        end else endif done \\
        \hline
        Instruction & [VarName] if while for print read &
        ; end else endif done \\
        \hline
        Assign & [VarName] &
        ; end else endif done \\
        \hline
        ExprArith & [VarName] [Number] - ( &
        ; end else endif done ) \\
        & & $=$ $>=$ $>$ $<=$ $<$ $<>$ \\
        & & and or then do \\
        & &  by to\\
        % & & or then do \\
        \hline
        ExprArithPrime & + - $\epsilon$ &
        Follow$^1$($<$ExprArith$>$) \\
        \hline
        ExprProd & [VarName] [Number] - ( &
        + - Follow$^1$($<$ExprArith$>$) \\
        \hline
        ExprProdPrime & * / $\epsilon$ &
        + - Follow$^1$($<$ExprArith$>$) \\
        \hline
        SumSubOp & + - & [VarName] [Number] - ( \\
        \hline
        ProdOp & * / & [VarName] [Number] - ( \\
        \hline
        If & if & ; end else endif done \\
        \hline
        Atom & [VarName] [Number] - ( &
        * / + - Follow$^1$($<$ExprArith$>$) \\
        \hline
        IfSeq & endif else & ; end else endif done \\
        \hline
        Cond & [VarName] [Number] - ( not & then do \\
        \hline
        CondPrime & or $\epsilon$ & then do \\
        \hline
        ConjCond & [VarName] [Number] - ( not &
        or then do  \\
        \hline
        ConjCondPrime & and $\epsilon$ & or then do\\
        \hline
        AtomCond & [VarName] [Number] - ( not &
        and or then do \\
        \hline
        SimpleCond & [VarName] [Number] - ( &
        and or then do \\
        \hline
        Comp & $=$ $>=$ $>$ $<=$ $<$ $<>$ &
        [VarName] [Number] - ( \\
        \hline
        While & while & ; end else endif done \\
        \hline
        For & for & ; end else endif done \\
        \hline
        ForOp & by $\epsilon$ & to \\
        \hline
        Print & print & ; end else endif done \\
        \hline
        Read & read & ; end else endif done \\
        \hline
    \end{tabular}
    \caption{First$^1$ and Follow$^1$ results}
    \label{fig:firstfollow}
\end{figure}
\clearpage
\restoregeometry
}

\paragraph{}

To be able to generate the action table of our LL(1) parser, it is necessary
to show the results of the First$^1(\cdot)$ and Follow$^1(\cdot)$
computation of each variables
belonging to the grammar. The results are shown in Figure \ref{fig:firstfollow}.

\subsection{Action table}

\label{explainactiontable}

\paragraph{}

Let $G<V, T, P, S>$ our transformed grammar.
Let the action table be $M(v, t)$ where
$v \in V$, and $t \in T$.

$M$ is filled in as follow:
\begin{itemize}
    \item $\forall v \in V$ and $\forall t \in$ First$^1(v)$,
    then set $M(v, t)$ as the rule number where $t$ is first.
    \item $\forall v \in V$ if $\epsilon \in $ First$^1(v)$ then
    $\forall t \in$ Follow$^1(v)$ set $M(v, t)$ as the rule number
    where $v \rightarrow \epsilon$.
    \item All others cell should remain empty.
\end{itemize}

The action table is available as a
\textit{comma separated} \texttt{.csv} file in
\textbf{doc/actionTable.csv}. This file only contains the
\textit{produce} rules.
It is not necessary to include the table that matches terminals together
since the only results possible is the action \textit{match}
for terminals that are the same.
In the case of the \texttt{end} terminal, the action is \textit{accept}.
An empty cell in the table induce a syntax error.
Section \ref{implactiontable} discuss in more details the implementation
of the action table.

\section{Scanner improvements}

\label{scanimprovements}
\paragraph{Read input}
In the first part of the project we were asked to implement a scanner.
To be able to use the scanner within the parser, we modified some part of the
scanner. The scanner is implemented in the abstract class \texttt{Scanner}.
The lexer generated class is named \texttt{GeneratedScanner} and will
implement the \texttt{Scanner}. To be able to read the input, \textit{i.e.}
the tokens, the method \texttt{Scanner.scan()} must be called, and
will return a list of \texttt{Symbol}. It is still possible to run the scanner
as a standalone application, see section \ref{howto}.
Figure \ref{fig:scannercall} shows an example of use of the scanner.

\begin{figure}[H]
    \begin{lstlisting}
            FileReader source = ImpCompilo.file("file.imp");
            Scanner scanner = new GeneratedScanner(source);
            List<Symbol> symbols = scanner.scan();
    \end{lstlisting}
    \caption{Scanner call to read the tokens from a source file}
    \label{fig:scannercall}
\end{figure}

\paragraph{Lexer improvements}

Two improvements were made in the lexer. (1) We modified the any ASCII
characters
regex \texttt{.} (dot) to regex that matches the complement of the
empty set \texttt{[\textasciicircum]}.
(2) Instead of printing \texttt{"Unkown symbol"}
and continue the scan when
a unrecognize symbol is detected, the scanner throws a
\texttt{UnknownTokenException} which logically stops the scan.



\section{LL(1) Parser implementation}

\subsection{Grammar and action table files}

\paragraph{Location of files}
The grammar and the action table \texttt{csv} files are stored
as resources in
\textbf{more/src/resources} directory.
\footnote{The csv file are \textit{comma separated}}.

\paragraph{Action table parsing}
To be able to run the parser it is required to be able to access
the action table to implement
the recursive descent LL(1) parser.
As mention in section \ref{explainactiontable}, the action table is stored
in a \texttt{csv} file \textbf{actionTable.csv}.
The first line of this file contains the
terminals, and all the other lines contains in the first column a variable
and the following columns either the rules to apply for each terminal.
If there is no rule \textit{i.e.} the cell is empty.

The method \texttt{LL1Parser.buildActionTable()} will read the file
to create a map that matches a variable and a terminal to a rule number
\textit{i.e.} a \texttt{Map<String, Map<String, Integer>>}.

\paragraph{Grammar parsing}
It is also necessary to access to set of rules of the grammar within the
parser. The grammar is stored in a \texttt{csv} file \textbf{grammar.csv}.
The first line of the file is the set of variables, and the second line
the set of terminals. All the following lines are the rules.
Since it is a \textit{Context-Free Grammar}, we know that the first element
of a rule is the left-hand side, and all the following element are the
the right-hand side.
The order of the rules is important since they will be numeroted in the parsing
order, and will be used to print the \textit{left-most derivation}.
The grammar in \textit{grammar.csv} is the same as shown in
Figure \ref{fig:fullcfg}.

The method \texttt{LL1Parser.buildGrammar()} will read the file to create
a map that matches a rule number to a list
of \textit{\{terminals, variables\}}
\textit{i.e.} \texttt{Map<Integer, List<String>>}.
A list could have been used, but a map was prefered since the first rule
is numeroted as 1.


The content of the list
of a rule is the following: the first element is the left-hand side, and
all the other elements are the right-hand side.
Note that when the right-hand side is $\epsilon$, the list of element
is just the left-hand side,
since there will be nothing to push on the stack for such
rules (see section \ref{implstack}).

\paragraph{}

This method of implementation of the grammar and the action table is very
interesting, since it allows to use the LL(1) parser for any LL(1) grammar,
by just changing the \texttt{csv} files.

\subsection{Complete action table}

\paragraph{}

The action table of an LL(1) parser in addition to the rule number of the
\textit{produce} action that is provided in the \texttt{csv} file must
also be able to provide the \textit{match} and \textit{accept} actions. It
is also necessary to provide a way to return a \textit{syntax error}
if none of these actions are possible. The method \texttt{LL1Parser.M(a, l)}
\footnote{The function name is taken from the Lecture Notes (p135)}
will return an \texttt{Integer} value, based on the parameters \texttt{a} and
\texttt{l},
 that is corresponding to either:
\begin{itemize}
    \item A rule number from the \texttt{csv} file if $a \in V \wedge l \in T$.
    If cell is empty in the file, then it returns the constant
    \texttt{SYNTAX\_ERROR}
    \item The constant \texttt{MATCH} if $(a, l \in T) \wedge (a = l)$
    \item The constant \texttt{ACCEPT} if $(a = l) \wedge (a =$ \texttt{end}$)$
    \item The constant \texttt{SYNTAX\_ERROR} for all other cases
\end{itemize}

\label{implactiontable}

\subsection{Explicit Stack}

\label{implstack}

\paragraph{PDA}


The implementation of the LL(1) parser is done by using a pushdown automaton.
We can actually avoid implementing any automaton states and only use
an explicit stack, since the LL(1) parse is a PDA with a single state
\footnote{Mention in the Lecture Notes page 135}.

\paragraph{Recursive descent}

The stack only
solution is a possible implementation of a recursive descent parser. Indeed,
by using recursion it uses the \textit{call stack} of subroutines.
The parser doesn't use the \textit{call stack} but
an explicit stack.

\paragraph{Implementation}
The parsing of the input is implemented in the method
\texttt{LL1Parser.parse()}. This method uses the stack and the scanned tokens
to run the parsing. The scanner throws a \texttt{UnknownTokenException}
if a unexpected token is detected.
The method also initialize the stack with the start symbol. The parsing
consists of a loop that runs an action from the action table \textit{i.e.}
the results of the method $M$, based on
the current top of stack and the current parsed token. The loop stops only
in two ways: (1) if the input is accepted \textit{i.e.} the action
\textit{accept} is ran; (2) a \textit{syntax error} is returned from
the action table.
The stack is implemented using the built-in class
\texttt{Stack} of \texttt{Java}.
Each action performs specific changes either on the stack, on the current
token or on the parsing process:
\begin{itemize}
    \item \textit{accept}: stops the parsing
    \item \textit{match}: pop the stack and read next token
    \item \textit{produce}: replace the top of stack by the right-hand side
    of the rule from the action table
    \item \textit{syntax error}: throws a \texttt{SyntaxError}
\end{itemize}

The implementation of the \texttt{parse}
method is in Figure \ref{fig:parsemethod} and the actions
in Figure \ref{fig:actionsmethod}.

\begin{figure}[H]
    \begin{center}
    \begin{lstlisting}
                stack.push(getStartSymbol());
                scannedTokens = scanner.scan();
                nextToken();
                parsing = token != null;
                while (parsing) {
                    Integer ruleNumber = M(stack.peek(), lexicalUnitOf(token));
                    switch (ruleNumber) {
                        case ACCEPT:
                            accept(token);
                            break;
                        case MATCH:
                            match(token);
                            break;
                        case SYNTAX_ERROR:
                            syntaxError(token);
                            break;
                        default:
                            produce(token, ruleNumber);
                            break;
                    }
                 }
\end{lstlisting}
    \end{center}
    \caption{\texttt{LL1Parser.parse()} method}
    \label{fig:parsemethod}
\end{figure}

\begin{figure}[H]
    \begin{subfigure}{.5\textwidth}
\begin{lstlisting}
        parsing = false;
\end{lstlisting}
        \caption{\textit{accept}}
        \label{fig:acceptaction}
    \end{subfigure}
    \begin{subfigure}{.5\textwidth}
\begin{lstlisting}
            stack.pop();
            nextToken();
\end{lstlisting}
        \caption{\textit{match}}
        \label{fig:matchaction}
    \end{subfigure}
    \begin{subfigure}{.5\textwidth}
\begin{lstlisting}
        stack.pop();
        pushRule(ruleNumber);
 \end{lstlisting}
        \caption{\textit{produce}}
        \label{fig:produceaction}
    \end{subfigure}
    \begin{subfigure}{.5\textwidth}

\begin{lstlisting}
            throw new SyntaxError(message);
\end{lstlisting}
        \caption{\textit{syntax error}}
        \label{fig:syntaxerroraction}
    \end{subfigure}
    \caption{Actions method}
    \label{fig:actionsmethod}
\end{figure}

\section{Project details}

\subsection{Project files}

\paragraph{Files} Java source files are in \textbf{more/src} directory.
The implementation of our compiler is divided in 4 parts:
\begin{enumerate}
    \item Files that composes the scanner: \texttt{Scanner.java} abstract
    class and the \texttt{LexicalAnalyzer.lex} that is used to generate
    the \texttt{GeneratedScanner.java} class, and
    the provided \texttt{Symbol.java} and \texttt{LexicalUnit.java} classes.
    \item Files that composes the parser: \texttt{LL1Parser.java}
    and the \texttt{TreeNode.java}, that is used to generate and print
    the derivation tree, classes.
    \item Utilities methods in \texttt{ImpCompilo.java} classes
    used by the scanner and the parser to read
    files, log informations or exit safely the compiler.
    \item A set of exceptions to be thrown by the scanner/parser.
\end{enumerate}

\paragraph{Javadoc} The JavaDoc is available in \textbf{doc/javadoc}.
The documentation for \texttt{Scanner.java} is \textbf{Scanner.html},
this document the scanner of part 1, including the improvements made
in section \ref{scanimprovements}.
The documentation for the parser
\texttt{LL1parser.java} is in
\textbf{LL1Parser.html}

\subsection{How to run}

\label{howto}

\paragraph{}

\begin{itemize}
    \item To run the parser use the following command:
    \begin{lstlisting}
        java -jar dist/part2.jar file.imp
    \end{lstlisting}
    This command will output the \textit{left-most derivation}.
    \item It is also possible to output the \textit{derivation tree}:
    \begin{lstlisting}
        java -jar dist/part2.jar file.imp printTree
    \end{lstlisting}
    \item It is still possible to run the scanner as a standalone appliction:
    \begin{lstlisting}
        java -jar dist/scanner.jar file.imp
    \end{lstlisting}
\end{itemize}


\section{Conclusion}

\end{document}
